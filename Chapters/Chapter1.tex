
\chapter{Introduction (\textbf{TODO})}
\label{Chapter1}

\section{Context and Motivations}

\section{Analysis of DNA sequences}

\subsection{Next-Generation Sequencing data}

\textbf{MAYBE MORE ORGANIC, WITHOUT SUBSECTIONS}

\subsection{Sequence Alignment}
The reads of a metagenome need to be compared to a set of reference genomes in order to estimate its composition. The first algorithms to perform the comparison of sequences were based on dynamic programming, but they soon became computationally unfeasible due to the fact that unforeseen amounts of data were being generated, that they only allowed pairwise comparison, and that their cost was quadratic in the size of the input sequences. In the current setting, the analysis of metagenomic reads with such a tool would require years.

Soon heuristic methods were developped for sequence comparison \dots

\subsection{Alignment-free methods}

\subsection{Estimation of Abundances}
